\documentclass{scrartcl}

\usepackage{parskip}
%\usepackage{fourier}

\title{My Master Thesis}
\author{Xun Zhu}

\begin{document}
\maketitle
\begin{abstract}
  PLOS has summarized a workflow\cite{???} to identify important miRs of
  pathways for a type of tumor, which has been proved very effective. We try to
  apply this kind of work flow on a new set of data about breast cancer aquired
  from TCGA. And found that it is indeed very effective. Notice that this new
  set of data has some interesting properties. It has very limited samples, so
  computing FDR in their original papaer becomes very no indicative. So instead
  we use another approach, to select the important miRs based on their permuted
  p-values.

  Also, we created three kinds of measurements trying to characterize the
  important miRs, which corresponds ot three matrices -- a binary matrix
  indicating whether an pathway is the target of miR, a integer matrix
  indicating the total number of binding sites between an miR-pathway pair, and
  a real number matrix indicating the "bind score" of an miR-pathway pair.
  These three measurements are expected to have an increasing accuracy, which
  turns out to be the case.

  Also, we compared the result (using this workflow) from two different source
  of target predictions: one from targetScan\cite{???}, and the other from
  miRanda\cite{???}.

\end{abstract}

\section{Introduction}

% Copy some intros from PLOS article, and mix in some wikipedia paragraphs.
% Mostly about ``what is mRNA'', ``why it is important'', etc.

\section{Data and tools}

\subsection{Collecting and processing data}

\subsubsection{TCGA}

% ask Sijia

\subsubsection{TargetScan}

% downloading, total up, rip off non-human

\subsubsection{MiRanda}

% downloading, ...

\subsection{R packages}

% R packages are very important because they can ...

\subsubsection{GSEA}

% same thing as below

\subsubsection{randomForest}

% package is written by ..., it provides ..., what are the parameters, how did
% you teak it, what are the expected consequnces, and are the results coincides
% with those expectations
% maybe generate a heat map?

\subsubsection{rfPermute}

% same thing here

\section{Workflow}

\subsection{Do the GSEA analysis}

% how did you prepare all the input files, 
% what are the running parameters
% << figure generated by GSEA >>
% how did you process the raw results generated
% how did you extract the leading edge genes
% how did you make them into a summary table

\subsection{Generate the matrix of score}

% three measurements of scores !!! what is the third one???

\subsection{Do the random forest algorithm}

% use randomForest first to get the importances
% then use rfPermute to get the permutation p-values
% << figure generated by rfPermute >>
% how many do you pick, what should be the criterion?

\subsection{Do the permutation p-value and selection}

% use the package, choose the cutoff, !!! why 0.01??? Statistical significance???

\section{Result Analysis}

\subsection{Comparison between the three measurements}

% The accuracy is expected to be increasing
% the result !!! ask prof. what would be a good way to compare the accuracy.
% well, are they seem correct to her?

\subsection{???}

% What are other analysis of the result? accuracy compare to clinical results?

\section{Appendix}

\subsection{Mathematica code}

% mostly in data processing

\end{document}
