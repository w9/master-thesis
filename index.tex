\documentclass{scrartcl}

\usepackage{parskip}

\title{My Master Thesis}

\begin{document}
\maketitle
\begin{abstruct}
  PLOS has summarized a workflow\cite{???} to identify important miRs of
  pathways for a type of tumor, which has been proved very effective. We try to
  apply this kind of work flow on a new set of data about breast cancer aquired
  from TCGA. And found that it is indeed very effective. Notice that this new
  set of data has some interesting properties. It has very limited samples, so
  computing FDR in their original papaer becomes very no indicative. So instead
  we use another approach, to select the important miRs based on their permuted
  p-values.

  Also, we created three kinds of measurements trying to characterize the important
  miRs, which corresponds ot three matrices -- a binary matrix indicating whether
  an pathway is the target of miR, a integer matrix indicating the total number
  of binding sites between an miR-pathway pair, and a real number matrix
  indicating the "bind score" of an miR-pathway pair. These three measurements are
  expected to have an increasing accuracy, which turns out to be the case.

  Also, we compared the result (using this workflow) from two different source
  of target predictions: one from targetScan\cite{???}, and the other from
  miRanda\cite{???}.

\end{abstruct}
\section{k}
\end{document}
