\documentclass{scrartcl}

\usepackage{parskip}
\usepackage[hidelinks]{hyperref}
\usepackage{booktabs}
\usepackage{fourier}
\usepackage{amsmath}
\usepackage{graphicx}
\usepackage{cite}

\numberwithin{figure}{section}
\numberwithin{table}{section}

\title{Bioinformatics strategies to predict important microRNAs determining
metastasizibility of breast cancer}
\author{Xun Zhu}
\date{}

\begin{document}
\maketitle
\begin{abstract}

  PLOS has summarized a workflow\cite{plos} to identify important miRs of
  pathways for a type of tumor, which uses GSEA\cite{subramanian2005gene} to
  measure the importance of each pathway. We try to apply this kind of workflow
  on a new set of data on breast cancer (aquired from TCGA). Because the number of
  samples is very limited, which is different from the situation in \cite{plos},
  instead of selecting miRs with high FDR, we instead selected the important miRs
  based on their permuted p-values.

  We created three kinds of measurements trying to characterize the
  important miRs, which corresponds to three matrices -- a binary matrix
  indicating whether an pathway is the target of miR, a integer matrix
  indicating the total number of binding sites between an miR-pathway pair, and
  a real number matrix indicating the "bind score" of an miR-pathway pair.
  We compared the selected important miRs from each measure and find some interesting
  results.

\end{abstract}

\bigskip
\textbf{Keywords.~~ } microRNA GSEA random-forest p-value

\clearpage
\tableofcontents
\clearpage

\section{Introduction}

% Copy some intros from PLOS article, and mix in some wikipedia paragraphs.
% Mostly about ``what is mRNA'', ``why it is important'', etc.
MicroRNAs (miRs) are small non-coding RNAs that interact with their gene target
coding mRNAs. Such small RNAs putatively inhibit translation by direct and
imperfect binding to the 3'- and 5'-untranslated regions
(UTR)\cite{bartel2004micrornas} and exert expression control with other
regulatory elements such as transcription factors \cite{martinez2009interplay,
shalgi2007global, wang2010transmir}.

The importance of miRs in the causations of cancers has been addressed in
various literatures including \cite{houbaviy2003embryonic, croce2009causes,
volinia2006microrna, lu2005microrna, esquela2006oncomirs, chen2005micrornas,
hurst2009metastamir, chan2011cancer}. An important question is how to
computationally find the important miRs in pathways, using the sample data of
differential expression of mRNAs, for which \cite{plos} provided a workflow
to use various tools and sources available to approch.

Here we apply this work flow to the data on the metastasis of breast cancer,
and try to find the important miRs in three different measure.

\section{Data and tools}

\subsection{Collecting and processing data}

\subsubsection{TCGA}

% How did we fetch the data?

Using the Cancer Genome Atlas
(TCGA\footnote{\url{http://cancergenome.nih.gov/}}), we utilized 12
metastasized breast cancer samples along with 9 non-metastasized control
samples that provided matching gene and miR expression profiles.

% how did you determine the ``differentially expressed miRs''? By Student's
% t-test?

\subsubsection{TargetScan}

TargetScan\footnote{\url{http://www.targetscan.org/}} is an online miR target
prediction service provided by Whitehead Institute for Biomedical Research. We
downloaded the lastest (up to Apr. 1st, 2014) version of their ``Summary
counts''
table\footnote{\url{http://www.targetscan.org/vert\_61/vert\_61\_data\_download/Summary\_Counts.txt.zip}},
which contains 18209042 pairs of miRs and mRNA and the number of different
types of binding sites. Because we only care about human mRNAs, we excluded all
the non-human pairs and the final table contains 5806825 pairs. We counted both
the conserved binding sites and the non-conserved binding sites.

% << a snippet of final table here >>
\begin{table}[h!]
\centering
\caption{A random sample of lines from the processed table of TargetScan}
\label{tab:ts}
\bigskip
\begin{tabular}{lll}
\toprule
  mRNA Gene Symbol & miRNA name      & \# Binding Sites \\
\midrule
  TMEM189          & hsa-miR-3934    & 2 \\
  PWWP2A           & hsa-miR-3646    & 1 \\
  F2RL3            & hsa-miR-3909    & 1 \\
  CSPG4            & hsa-miR-331-3p  & 2 \\
  MAPK14           & hsa-miR-4459    & 1 \\
  MYO3B            & hsa-miR-3675-5p & 1 \\
  RREB1            & hsa-miR-205     & 1 \\
  PGF              & hsa-miR-4691-5p & 1 \\
  ZC3H8            & hsa-miR-512-5p  & 1 \\
  C15orf38-AP3S2   & hsa-miR-3115    & 1 \\
  \ldots           & \ldots          & \ldots \\
\bottomrule
\end{tabular}
\end{table}

\subsubsection{MiRanda}

MiRanda\footnote{\url{http://www.microrna.org/}} is another online miR target
prediction service. It is provided by Memorial Sloan Kettering Cancer
Center\footnote{\url{http://www.mskcc.org/}}. We downloaded the lastest (up to
Apr. 1st, 2014) version of their ``Target Site
Predictions''\footnote{\url{http://www.microrna.org/microrna/getDownloads.do}}
with all four categories (regardless of good/bad mirSVR score,
conserve/non-conserve). We combined the four tables and sumed up the counts.
% why all of them???
The total number of pairs after combining and totaling is 7333127.

It's worth noting that miRanda uses both the ``-3p/-5p'' notation and the
deprecated\footnote{The deprecation is explained in
\url{http://www.mirbase.org/blog/2011/04/whats-in-a-name/}} ``star-notation''
(*) to indicate minor product, for example, in Table~\ref{tab:mir}, we see that
it uses ``hsa-miR-545*'' to indicate a minor product of hsa-mir-545. Therefore
there are some name-wise inconsistency with targetScan.

\begin{table}[h!]
\centering
\caption{A random sample of lines from the processed table of MiRanda}
\label{tab:mir}
\begin{tabular}{lll}
\toprule
  mRNA Gene Symbol & miRNA name     & \# Binding Sites \\
\midrule
  JAM3             & hsa-miR-3173   & 1 \\
  SNPH             & hsa-miR-502-3p & 2 \\
  FAM126A          & hsa-miR-874    & 2 \\
  ARGFX            & hsa-miR-4280   & 1 \\
  GNE              & hsa-miR-3153   & 1 \\
  BEND2            & hsa-miR-2052   & 2 \\
  ASCC1            & hsa-miR-188-5p & 3 \\
  TSC22D2          & hsa-miR-375    & 9 \\
  UACA             & hsa-miR-377    & 1 \\
  TMF1             & hsa-miR-545*   & 3 \\
  \ldots           & \ldots         & \ldots \\
\bottomrule
\end{tabular}
\end{table}

\subsection{R packages}

% R packages are very important because they can ...

\subsubsection{GSEA}

Gene Set Enrichment Analysis (GSEA) is develop at the Broad Institute of MIT
and Harvard.  We used the lasted version (1.0) of their R
package\footnote{\url{http://www.broadinstitute.org/gsea/msigdb/download\_file.jsp?filePath=/resources/software/GSEA-P-R.1.0.zip}
(might need to log in)}. We selected only the top 20 scoring gene sets for
analysis. We used the ``C2'' version of gene set database, and used 1000 as the
number of random permutations. The other parameters we used can be found in the
Appendix~\ref{src:gsea}.

\subsubsection{randomForest}

% package is written by ..., it provides ..., what are the parameters, how did
% you teak it, what are the expected consequnces, and are the results coincides
% with those expectations
% maybe generate a heat map?

Random Forest is an R package for application of random forest algorithm. It is
developed by Leo Breiman and Adele Cutler. We used its current version (4.6-7).
The parameters we used can be found in the Appendix~\ref{src:rf}.

\subsubsection{rfPermute}

RfPermute is an R package that uses the result from Random Forest package and
compute the significance of importance by permuting the response variable. It
is developed by Eric
Archer\footnote{\url{http://cran.r-project.org/web/packages/rfPermute/index.html}}.
We use this package to verify the significance of the result. From statistical
convention, we set the cut-off to be $p<0.01$.

\section{Workflow}

Figure~\ref{fig:wf} demonstrates the slightly modified workflow from PLOS\cite{plos}.

\begin{figure}[h!]
  \caption{The modified workflow to select important miRs from TCGA data}
  \label{fig:wf}
  \bigskip
  \includegraphics[width=\textwidth]{{workflow}.pdf}
\end{figure}

\subsection{Performe GSEA analysis}

% how did you prepare all the input files,
% what are the running parameters
% << figure generated by GSEA >>
% how did you process the raw results generated
% how did you extract the leading edge genes
% how did you make them into a summary table

In order to feed the data into GSEA-R, we need to first process the raw tables
downloaded from TCGA. We wrote a script to combine the two tables, create a
corresponding list of phenotypes, and add the necessary arguments in the beginning
of the table.

GSEA-R selected 40 pathways representing the most contributing to the change of
phenotype from 522 pathways stored in C2 database. 20 for scoring highest in
metastasized group and another 20 for scoring lowest (highest absolute value)
in primary group (non-metastasized control group). Table~\ref{tab:gs1} and
Table~\ref{tab:gs2} list the names of pathways selected and their corresponding
scores. Figure~\ref{fig:gsea} lists the heat map, p-values and other information produced by
GSEA-R.

GSEA-R also computes the leading edges genes for each pathway. Since leading
edge genes statistically represents the most important mRNAs in a pathway.
We proceed the following steps using only the leading edge genes in each
pathway. This is also the approach used in \cite{plos}.

\begin{table}[h!]
\centering
\caption{The 20 pathways in metastasized group selected by GSEA-R}
\label{tab:gs1}
\bigskip
\begin{tabular}{ll}
\toprule
  Pathway Names                                             & Score   \\   
\midrule
  P53\_UP                                                   & 1.4027  \\
  MAP00480\_Glutathione\_metabolism                         & 1.3638  \\
  intrinsicPathway                                          & 1.283   \\
  MAP00360\_Phenylalanine\_metabolism                       & 1.2174  \\
  Matrix\_Metalloproteinases                                & 1.2056  \\
  MAP00340\_Histidine\_metabolism                           & 1.195   \\
  HOX\_LIST\_JP                                             & 1.1591  \\
  ADULT\_LIVER\_vs\_FETAL\_LIVER\_GNF2                      & 1.1514  \\
  electron\_transporter\_activity                           & 1.083   \\
  no1Pathway                                                & 1.0657  \\
  MAP00350\_Tyrosine\_metabolism                            & 1.0628  \\
  GO\_ROS                                                   & 1.0484  \\
  MAP00220\_Urea\_cycle\_and\_metabolism\_of\_amino\_groups & 1.0259  \\
  FRASOR\_ER\_UP                                            & 0.95187 \\
  ST\_Wnt\_beta\_catenin\_Pathway                           & 0.93623 \\
  p53hypoxiaPathway                                         & 0.92514 \\
  SIG\_CD40PATHWAYMAP                                       & 0.90249 \\
  AR\_MOUSE\_PLUS\_TESTO\_FROM\_NETAFFX                     & 0.89128 \\
  AR\_ORTHOS\_MAPPED\_TO\_U133\_VIA\_NETAFFX                & 0.88948 \\
  AR\_MOUSE                                                 & 0.88948 \\
\bottomrule
\end{tabular}
\end{table}

\begin{table}[h!]
\centering
\caption{The 20 pathways in primary group selected by GSEA-R}
\label{tab:gs2}
\bigskip
\begin{tabular}{ll}
\toprule
  Pathway Names                                             & Score   \\   
\midrule
  shh\_lisa                                                 & -1.7159 \\
  SA\_CASPASE\_CASCADE                                      & -1.7064 \\
  XINACT\_MERGED                                            & -1.6725 \\
  41bbPathway                                               & -1.6581 \\
  tnfr1Pathway                                              & -1.6289 \\
  MAP00252\_Alanine\_and\_aspartate\_metabolism             & -1.5879 \\
  CR\_CELL\_CYCLE                                           & -1.5836 \\
  Cell\_Cycle                                               & -1.5798 \\
  CR\_REPAIR                                                & -1.5732 \\
  fasPathway                                                & -1.573  \\
  g2Pathway                                                 & -1.5727 \\
  GLUCOSE\_DOWN                                             & -1.5652 \\
  il7Pathway                                                & -1.5594 \\
  chrebpPathway                                             & -1.5504 \\
  il1rPathway                                               & -1.5485 \\
  MAPK\_Cascade                                             & -1.5191 \\
  atrbrcaPathway                                            & -1.5186 \\
  Il12Pathway                                               & -1.5129 \\
  mitochondriaPathway                                       & -1.5096 \\
  ST\_Fas\_Signaling\_Pathway                               & -1.5035 \\
\bottomrule
\end{tabular}
\end{table}

\begin{figure}[h!]
  \centering
  \caption{The summary of GSEA-R}
  \label{fig:gsea}
  \includegraphics[width=\textwidth]{{breast_cancer.global.plots}.pdf}
\end{figure}

\subsection{Post-GSEA-R processing}

Before we proceed to generate the matrix of score, we need to first
aggregate the data from the output of GSEA-R. We wrote a script to
generate a summary table of pathways and their corresponding scores,
along with the their corresponding leading edge genes (LEGs). A snippet
of the summary can be found in Table~\ref{tab:legs}.

\begin{table}[h!]
\centering
\caption{Pathways, scores and leading edge genes}
\label{tab:legs}
\bigskip
\scriptsize
\begin{tabular}{lllllll}
\toprule
  Pathway Name                                              & Score   & \multicolumn{4}{l}{First Few Leading Edge Genes} & \dots \\
\midrule
  P53\_UP                                                   & 1.4027  & NDN      & IGFBP6  & FHL2    & APLP1   & \dots \\
  MAP00480\_Glutathione\_metabolism                         & 1.3638  & GGT1     & GPX4    & G6PD    & GSTM5   & \dots \\
  intrinsicPathway                                          & 1.283   & F10      & PROC    & F2      & F12     & \dots \\
  MAP00360\_Phenylalanine\_metabolism                       & 1.2174  & TAT      & ALDH3A1 & ABP1    & MAOA    & \dots \\
  Matrix\_Metalloproteinases                                & 1.2056  & MMP17    & MMP28   & BSG     & TIMP3   & \dots \\
  MAP00340\_Histidine\_metabolism                           & 1.195   & ALDH3A1  & ABP1    & MAOA    & AOC3    & \dots \\
  HOX\_LIST\_JP                                             & 1.1591  & HOXB7    & HOXD1   & HOXD9   & HOXB2   & \dots \\
  ADULT\_LIVER\_vs\_FETAL\_LIVER\_GNF2                      & 1.1514  & ADH1C    & SIGIRR  & CES2    & RARRES2 & \dots \\
  electron\_transporter\_activity                           & 1.083   & SPR      & ETFB    & ADH1C   & BLVRA   & \dots \\
  no1Pathway                                                & 1.0657  & FLT4     & BDKRB2  & NOS3    & CAV1    & \dots \\
  MAP00350\_Tyrosine\_metabolism                            & 1.0628  & ADH1C    & HGD     & TAT     & ALDH3A1 & \dots \\
  GO\_ROS                                                   & 1.0484  & CCS      & PDLIM1  & PRDX2   & MTL5    & \dots \\
  MAP00220\_Urea\_cycle\_and\_metabolism\_of \dots          & 1.0259  & GAMT     & ARG1    & GLUD1   & OTC     & \dots \\
  FRASOR\_ER\_UP                                            & 0.95187 & IGFBP4   & SLC39A6 & AREG    & GLRB    & \dots \\
  ST\_Wnt\_beta\_catenin\_Pathway                           & 0.93623 & TSHB     & NKD2    & DKK4    & PIN1    & \dots \\
  p53hypoxiaPathway                                         & 0.92514 & FHL2     & GADD45A & CPB2    & CDKN1A  & \dots \\
  SIG\_CD40PATHWAYMAP                                       & 0.90249 & IKBKG    & MAPK11  & NFKBIL1 & MAPK3   & \dots \\
  AR\_MOUSE\_PLUS\_TESTO\_FROM\_NETAFFX                     & 0.89128 & GPC1     & ADRA2C  & CYP3A5  & RAMP3   & \dots \\
  AR\_ORTHOS\_MAPPED\_TO\_U133\_VIA\_NETAFFX                & 0.88948 & GPC1     & ADRA2C  & RAMP3   & CA4     & \dots \\
  AR\_MOUSE                                                 & 0.88948 & GPC1     & ADRA2C  & RAMP3   & CA4     & \dots \\
\midrule
  shh\_lisa                                                 & -1.7159 & XPO1     & DYRK1A  & CDK2    & CDK8    & \dots \\
  SA\_CASPASE\_CASCADE                                      & -1.7064 & APAF1    & GZMB    & CASP3   & DFFB    & \dots \\
  XINACT\_MERGED                                            & -1.6725 & EIF1AX   & USP9X   & PRKX    & ATP6AP2 & \dots \\
  41bbPathway                                               & -1.6581 & MAPK8    & TNFRSF9 & CHUK    & MAPK14  & \dots \\
  tnfr1Pathway                                              & -1.6289 & MAP3K7   & CASP3   & MAPK8   & LMNB1   & \dots \\
  MAP00252\_Alanine\_and\_aspartate\_metabolism             & -1.5879 & NARS     & DDX3X   & ADSL    & ADSS    & \dots \\
  CR\_CELL\_CYCLE                                           & -1.5836 & FRK      & TTK     & SKP2    & CDK6    & \dots \\
  Cell\_Cycle                                               & -1.5798 & SKP2     & CDK6    & HDAC2   & CHEK2   & \dots \\
  CR\_REPAIR                                                & -1.5732 & BRCA2    & CHEK2   & PMS1    & RAD1    & \dots \\
  fasPathway                                                & -1.573  & MAP3K7   & CASP3   & MAPK8   & LMNB1   & \dots \\
  g2Pathway                                                 & -1.5727 & CHEK2    & EP300   & CDC25A  & ATR     & \dots \\
  GLUCOSE\_DOWN                                             & -1.5652 & ZFR      & DEK     & KIF11   & PAPSS1  & \dots \\
  il7Pathway                                                & -1.5594 & IL7      & JAK1    & PIK3CA  & EP300   & \dots \\
  chrebpPathway                                             & -1.5504 & PRKAA1   & GNB1    & PRKAG2  & PRKAR2A & \dots \\
  il1rPathway                                               & -1.5485 & MAP3K7   & MAPK8   & IL1A    & TNF     & \dots \\
  MAPK\_Cascade                                             & -1.5191 & NRAS     & BRAF    & MAPK1   & RAF1    & \dots \\
  atrbrcaPathway                                            & -1.5186 & FANCD2   & BRCA2   & CHEK2   & RAD1    & \dots \\
  Il12Pathway                                               & -1.5129 & JAK2     & MAPK8   & CCR5    & MAPK14  & \dots \\
  mitochondriaPathway                                       & -1.5096 & APAF1    & CASP3   & DFFB    & BIRC3   & \dots \\
  ST\_Fas\_Signaling\_Pathway                               & -1.5035 & CASP8AP2 & NFAT5   & ROCK1   & CASP3   & \dots \\
\bottomrule
\end{tabular}
\end{table}

% three measurements of scores !!! what is the third one???
% ??? idea: how about the score of each pathway generated by GSEA?

\subsection{Generate the matrix of score}

A matrix of score is the training data for random forest. In our case, it is a
matrix with 40 rows (each representing a pathway selected by GSEA-R) and as
many columns as there are miRs in the data given by the source.  Here we take
targetScan for a example. We will be generating a $40\times1525$ matrix, which
will be abbreviated as matrix-2.  Table~\ref{tab:mat2} demonstrates a snippet
of the matrix generated.

As a comparison, we also generate another two matrices: one with less
information used, and one with more information used. The former is a binary
matrix to indicate merely whether there is at least one binding site between a
pair. A snippet of this table can be found in Table~\ref{tab:mat1}. This matrix
will be abbreviated as matrix-1.  The latter is matrix-2 with each entry timed
by the differential expression of corresponding miR, the data of which also
comes from TCGA. This matrix will be abbreviated as matrix-3.  Notice we number
the matrices according to the amount of information they used.

\begin{table}[h!]
  \centering
  \caption{Snippet of the binary matrix (matrix-1}
  \label{tab:mat1}
\smallskip
  \small
  \begin{tabular}{lllllllllllllll}
    \toprule
    & \rotatebox{90}{hsa-miR-1282} & \rotatebox{90}{hsa-miR-137} & \rotatebox{90}{hsa-miR-3117-5p} & \rotatebox{90}{hsa-miR-32} & \rotatebox{90}{hsa-miR-3673} & \rotatebox{90}{hsa-miR-3976} & \rotatebox{90}{hsa-miR-4428} & \rotatebox{90}{hsa-miR-4522} & \dots \\
    \midrule                                                   
    P53\_UP                                          & 0            & 1           & 1               & 1          & 1            & 1            & 1            & 1 & \dots \\
    MAP00480\_Glutathione\_metabolism                & 0            & 0           & 0               & 0          & 1            & 0            & 1            & 1 & \dots \\
    intrinsicPathway                                 & 0            & 1           & 1               & 0          & 1            & 1            & 1            & 1 & \dots \\
    MAP00360\_Phenylalanine\_metabolism              & 0            & 0           & 1               & 0          & 1            & 1            & 1            & 0 & \dots \\
    Matrix\_Metalloproteinases                       & 0            & 0           & 1               & 1          & 1            & 1            & 1            & 1 & \dots \\
    MAP00340\_Histidine\_metabolism                  & 0            & 1           & 1               & 0          & 1            & 1            & 1            & 0 & \dots \\
    HOX\_LIST\_JP                                    & 0            & 1           & 0               & 1          & 1            & 1            & 1            & 0 & \dots \\
    ADULT\_LIVER\_vs\_FETAL\_LIVER\_GNF2             & 0            & 1           & 1               & 1          & 1            & 1            & 1            & 1 & \dots \\
    electron\_transporter\_activity                  & 0            & 1           & 1               & 1          & 1            & 1            & 1            & 1 & \dots \\
    no1Pathway                                       & 1            & 1           & 1               & 1          & 1            & 0            & 1            & 1 & \dots \\
    MAP00350\_Tyrosine\_metabolism                   & 0            & 1           & 1               & 0          & 1            & 1            & 0            & 0 & \dots \\
    GO\_ROS                                          & 0            & 1           & 1               & 0          & 1            & 1            & 1            & 0 & \dots \\
    MAP00220\_Urea\_cycle\_and\_metabolism\_of \dots & 0            & 0           & 1               & 0          & 1            & 0            & 1            & 1 & \dots \\
    FRASOR\_ER\_UP                                   & 1            & 1           & 1               & 1          & 1            & 1            & 1            & 1 & \dots \\
    ST\_Wnt\_beta\_catenin\_Pathway                  & 0            & 1           & 0               & 1          & 1            & 0            & 1            & 1 & \dots \\
    p53hypoxiaPathway                                & 0            & 1           & 0               & 1          & 0            & 0            & 1            & 0 & \dots \\
    SIG\_CD40PATHWAYMAP                              & 0            & 0           & 1               & 0          & 0            & 0            & 1            & 0 & \dots \\
    AR\_MOUSE\_PLUS\_TESTO\_FROM\_NETAFFX            & 0            & 0           & 1               & 0          & 1            & 0            & 1            & 1 & \dots \\
    AR\_ORTHOS\_MAPPED\_TO\_U133\_VIA\_NETAFFX       & 0            & 0           & 1               & 0          & 1            & 0            & 1            & 1 & \dots \\
    AR\_MOUSE                                        & 0            & 0           & 1               & 0          & 1            & 0            & 1            & 1 & \dots \\
    shh\_lisa                                        & 1            & 1           & 1               & 1          & 1            & 1            & 1            & 1 & \dots \\
    SA\_CASPASE\_CASCADE                             & 0            & 1           & 1               & 0          & 0            & 0            & 1            & 1 & \dots \\
    XINACT\_MERGED                                   & 0            & 1           & 1               & 0          & 1            & 1            & 1            & 1 & \dots \\
    41bbPathway                                      & 0            & 1           & 1               & 1          & 1            & 1            & 1            & 1 & \dots \\
    tnfr1Pathway                                     & 0            & 1           & 1               & 0          & 1            & 1            & 1            & 1 & \dots \\
    MAP00252\_Alanine\_and\_aspartate\_metabolism    & 0            & 1           & 1               & 0          & 1            & 0            & 1            & 0 & \dots \\
    CR\_CELL\_CYCLE                                  & 1            & 1           & 1               & 1          & 1            & 1            & 1            & 1 & \dots \\
    Cell\_Cycle                                      & 1            & 1           & 1               & 1          & 1            & 1            & 1            & 1 & \dots \\
    CR\_REPAIR                                       & 0            & 1           & 1               & 0          & 1            & 1            & 1            & 1 & \dots \\
    fasPathway                                       & 0            & 1           & 1               & 0          & 1            & 1            & 1            & 1 & \dots \\
    g2Pathway                                        & 1            & 0           & 1               & 1          & 1            & 1            & 1            & 1 & \dots \\
    GLUCOSE\_DOWN                                    & 1            & 1           & 1               & 1          & 1            & 1            & 2            & 1 & \dots \\
    il7Pathway                                       & 0            & 1           & 0               & 1          & 0            & 1            & 1            & 1 & \dots \\
    chrebpPathway                                    & 1            & 1           & 1               & 1          & 1            & 1            & 1            & 0 & \dots \\
    il1rPathway                                      & 0            & 1           & 1               & 0          & 1            & 1            & 1            & 1 & \dots \\
    MAPK\_Cascade                                    & 1            & 1           & 1               & 1          & 1            & 1            & 1            & 1 & \dots \\
    atrbrcaPathway                                   & 1            & 1           & 1               & 1          & 1            & 0            & 1            & 1 & \dots \\
    Il12Pathway                                      & 0            & 1           & 0               & 0          & 0            & 1            & 1            & 1 & \dots \\
    mitochondriaPathway                              & 0            & 1           & 1               & 0          & 1            & 1            & 1            & 1 & \dots \\
    ST\_Fas\_Signaling\_Pathway                      & 1            & 1           & 1               & 0          & 1            & 1            & 1            & 1 & \dots \\
    \bottomrule
  \end{tabular}
\end{table}

\begin{table}[h!]
\centering
\caption{Snippet of the generated binding site counting matrix (matrix-2)}
\label{tab:mat2}
\smallskip
\small
\begin{tabular}{lllllllllllllll}
\toprule
& \rotatebox{90}{hsa-miR-1282} & \rotatebox{90}{hsa-miR-137} & \rotatebox{90}{hsa-miR-3117-5p} & \rotatebox{90}{hsa-miR-32} & \rotatebox{90}{hsa-miR-3673} & \rotatebox{90}{hsa-miR-3976} & \rotatebox{90}{hsa-miR-4428} & \rotatebox{90}{hsa-miR-4522} & \dots \\
\midrule                                                   
P53\_UP                                          & 0            & 1           & 1               & 2          & 2            & 3            & 3            & 1 & \dots \\
MAP00480\_Glutathione\_metabolism                & 0            & 0           & 0               & 0          & 1            & 0            & 1            & 2 & \dots \\
intrinsicPathway                                 & 0            & 2           & 1               & 0          & 6            & 1            & 2            & 2 & \dots \\
MAP00360\_Phenylalanine\_metabolism              & 0            & 0           & 1               & 0          & 2            & 2            & 2            & 0 & \dots \\
Matrix\_Metalloproteinases                       & 0            & 0           & 3               & 1          & 4            & 4            & 2            & 1 & \dots \\
MAP00340\_Histidine\_metabolism                  & 0            & 1           & 1               & 0          & 2            & 2            & 1            & 0 & \dots \\
HOX\_LIST\_JP                                    & 0            & 1           & 0               & 3          & 3            & 2            & 2            & 0 & \dots \\
ADULT\_LIVER\_vs\_FETAL\_LIVER\_GNF2             & 0            & 4           & 3               & 2          & 5            & 2            & 3            & 4 & \dots \\
electron\_transporter\_activity                  & 0            & 4           & 2               & 2          & 3            & 7            & 6            & 4 & \dots \\
no1Pathway                                       & 2            & 1           & 2               & 1          & 1            & 0            & 2            & 1 & \dots \\
MAP00350\_Tyrosine\_metabolism                   & 0            & 1           & 1               & 0          & 2            & 2            & 0            & 0 & \dots \\
GO\_ROS                                          & 0            & 2           & 1               & 0          & 1            & 2            & 5            & 0 & \dots \\
MAP00220\_Urea\_cycle\_and\_metabolism\_of \dots & 0            & 0           & 1               & 0          & 1            & 0            & 2            & 1 & \dots \\
FRASOR\_ER\_UP                                   & 1            & 3           & 4               & 1          & 2            & 2            & 7            & 2 & \dots \\
ST\_Wnt\_beta\_catenin\_Pathway                  & 0            & 2           & 0               & 2          & 1            & 0            & 1            & 1 & \dots \\
p53hypoxiaPathway                                & 0            & 2           & 0               & 2          & 0            & 0            & 1            & 0 & \dots \\
SIG\_CD40PATHWAYMAP                              & 0            & 0           & 1               & 0          & 0            & 0            & 3            & 0 & \dots \\
AR\_MOUSE\_PLUS\_TESTO\_FROM\_NETAFFX            & 0            & 0           & 3               & 0          & 1            & 0            & 2            & 1 & \dots \\
AR\_ORTHOS\_MAPPED\_TO\_U133\_VIA\_NETAFFX       & 0            & 0           & 3               & 0          & 1            & 0            & 2            & 1 & \dots \\
AR\_MOUSE                                        & 0            & 0           & 3               & 0          & 1            & 0            & 2            & 1 & \dots \\
shh\_lisa                                        & 1            & 8           & 2               & 3          & 3            & 1            & 2            & 1 & \dots \\
SA\_CASPASE\_CASCADE                             & 0            & 4           & 2               & 0          & 0            & 0            & 3            & 2 & \dots \\
XINACT\_MERGED                                   & 0            & 4           & 1               & 0          & 13           & 3            & 5            & 4 & \dots \\
41bbPathway                                      & 0            & 3           & 2               & 1          & 2            & 2            & 1            & 1 & \dots \\
tnfr1Pathway                                     & 0            & 7           & 2               & 0          & 4            & 3            & 6            & 4 & \dots \\
MAP00252\_Alanine\_and\_aspartate\_metabolism    & 0            & 2           & 2               & 0          & 1            & 0            & 2            & 0 & \dots \\
CR\_CELL\_CYCLE                                  & 4            & 6           & 3               & 3          & 11           & 6            & 7            & 5 & \dots \\
Cell\_Cycle                                      & 2            & 6           & 7               & 5          & 14           & 9            & 10           & 5 & \dots \\
CR\_REPAIR                                       & 0            & 3           & 1               & 0          & 5            & 2            & 4            & 6 & \dots \\
fasPathway                                       & 0            & 7           & 2               & 0          & 5            & 5            & 8            & 4 & \dots \\
g2Pathway                                        & 1            & 0           & 2               & 2          & 6            & 1            & 4            & 3 & \dots \\
GLUCOSE\_DOWN                                    & 1            & 13          & 8               & 2          & 14           & 4            & 20           & 13& \dots \\
il7Pathway                                       & 0            & 1           & 0               & 1          & 0            & 1            & 2            & 1 & \dots \\
chrebpPathway                                    & 2            & 4           & 1               & 1          & 1            & 1            & 1            & 0 & \dots \\
il1rPathway                                      & 0            & 3           & 3               & 0          & 4            & 5            & 4            & 5 & \dots \\
MAPK\_Cascade                                    & 3            & 7           & 3               & 1          & 3            & 6            & 3            & 3 & \dots \\
atrbrcaPathway                                   & 1            & 3           & 2               & 2          & 5            & 0            & 5            & 5 & \dots \\
Il12Pathway                                      & 0            & 3           & 0               & 0          & 0            & 1            & 2            & 1 & \dots \\
mitochondriaPathway                              & 0            & 4           & 2               & 0          & 1            & 2            & 4            & 3 & \dots \\
ST\_Fas\_Signaling\_Pathway                      & 1            & 6           & 3               & 0          & 7            & 2            & 8            & 3 & \dots \\
\bottomrule
\end{tabular}
\end{table}

\clearpage

\subsection{Apply the random forest algorithm}

% use randomForest first to get the importances then use rfPermute to get the
% permutation p-values << figure generated by rfPermute >> how many do you
% pick, what should be the criterion?
Now we can feed the generated matrix into the random forest R package, and it
computes an importance score for each miR, indicating the extend to which the
presense (in the case of matrix-1) or number of binding sites (in the case of
matrix-2) of that miR affects the score of a pathway. Since the affect can be 
on both directions, we care about not only those with with the largest positive
importance, but also those those with the largest negative importance.

Here we feed matrix-2 as an example. A list of top and bottom 10 importance
miRs can be found in Table~\ref{tab:imp}. As we can see from the table, the
negative miRs has much smaller importance in terms of magnitude. Thus we
consider not to count them as important.

\begin{table}[h!]
  \centering
  \caption{The list of miRs with the most postive and negative importance}
  \label{tab:imp}
\bigskip
  \begin{tabular}{llll}
    \toprule
    Positive miRs   & Importance & Negative miRs  & Importance   \\
    \midrule
    hsa.miR.3613.3p & 0.37676027 & hsa.miR.3188   & -0.004386791 \\
    hsa.miR.664     & 0.25072487 & hsa.miR.613    & -0.003706353 \\
    hsa.miR.217     & 0.19105828 & hsa.miR.374b   & -0.003467034 \\
    hsa.miR.600     & 0.13873220 & hsa.miR.4748   & -0.003281420 \\
    hsa.miR.203     & 0.11067679 & hsa.miR.628.5p & -0.003214184 \\
    hsa.miR.498     & 0.09804174 & hsa.miR.761    & -0.003153555 \\
    hsa.miR.4282    & 0.05898798 & hsa.miR.938    & -0.003113474 \\
    hsa.miR.579     & 0.04513470 & hsa.miR.377    & -0.002981766 \\
    hsa.miR.216b    & 0.03228265 & hsa.miR.4517   & -0.002953550 \\
    hsa.miR.1208    & 0.02719592 & hsa.miR.3175   & -0.002895210 \\
    \bottomrule
  \end{tabular}
\end{table}

\subsection{Compute the permutation p-value and selection}

% use the package, choose the cutoff, !!! why 0.01??? Statistical significance???
To verify the statistical significance of the result from random forest,
we need to compute the p-value by randomly permuting the enrichment scores and
the matrix entries. We do this by utilizing the rfPermute R package.
We repeated the process 100 times and constructed null-distributions of
randomized importance scores for each miR/pathway pair. Fitting such distributions
with a Z-test, we may calculated p-values. The miRs with the smallest p-values
are listed in the Table~\ref{tab:pval}. Note that the p-values are close to
each other in the table because we have very limited sample, much less than
the number of miRs we are trying to examine, this makes so that the final p-value
can only take a limited set of values, therefore it's not a surprise that
the least ones take the very same value.

\begin{table}[h!]
  \centering
  \caption{miRs with the smallest p-values}
  \label{tab:pval}
\bigskip
  \begin{tabular}{llllll}
    \toprule
    miR             & p-value    \\
    \midrule
    hsa.miR.1297    & 0.00990099 \\
    hsa.miR.1323    & 0.00990099 \\
    hsa.miR.190     & 0.00990099 \\
    hsa.miR.203     & 0.00990099 \\
    hsa.miR.216b    & 0.00990099 \\
    hsa.miR.217     & 0.00990099 \\
    hsa.miR.3120.5p & 0.00990099 \\
    hsa.miR.323.3p  & 0.00990099 \\
    hsa.miR.3529    & 0.00990099 \\
    hsa.miR.3613.3p & 0.00990099 \\
    \bottomrule
  \end{tabular}
\end{table}

\section{Result Analysis}

% The accuracy is expected to be increasing the result ??? ask prof. what would
% be a good way to compare the accuracy.  well, are they seem correct to her?
We select in each measure those with the p-value $<0.01$ as the important miRs
in pathways.  The selected list can be found in Table~\ref{tab:sel}, from which
we may draw a venn diagram, as shown in Figure~\ref{fig:venn}

As we can see from the venn diagram, the important miRs generated from matrix-2
and matrix-3 have 4 overlapping miRs, while matrix-1 has only 1 overlapping with
matrix-2 and 2 with matrix-3. However, this is expected since matrix-1 is
binary and uses significantly less information than the other two, thus its
results don't quite agree with those from the other two. In fact, as we can see
there is no three way agreement among the three measures.

\begin{table}[h!]
  \centering
  \caption{The selected miRs by each measure}
  \label{tab:sel}
\bigskip
  \begin{tabular}{lll}
    \toprule
    Matrix-1     & Matrix-2     & Matrix-3     \\
    {\small(Binary)} & {\small(\#bdg. sites)} & {\small(\#bdg. sites $\times$ diff. expr.)} \\
    \midrule
    hsa-miR-1208 & hsa-miR-1258 & hsa-miR-1246 \\
    hsa-miR-203  & hsa-miR-1278 & hsa-miR-1286 \\
    hsa-miR-2116 & hsa-miR-1471 & hsa-miR-1290 \\
    hsa-miR-217  & hsa-miR-222  & hsa-miR-1322 \\
    hsa-miR-3163 & hsa-miR-23c  & hsa-miR-1913 \\
    hsa-miR-3686 & hsa-miR-3713 & hsa-miR-2113 \\
    hsa-miR-3908 & hsa-miR-376c & hsa-miR-3125 \\
    hsa-miR-4282 & hsa-miR-4292 & hsa-miR-3143 \\
    hsa-miR-548t & hsa-miR-4305 & hsa-miR-3148 \\
    hsa-miR-579  & hsa-miR-4325 & hsa-miR-3658 \\
    hsa-miR-600  & hsa-miR-549  & hsa-miR-3685 \\
    hsa-miR-664  & hsa-miR-562  & hsa-miR-3692 \\
    hsa-miR-944  & hsa-miR-586  & hsa-miR-4254 \\
    hsa-miR-374c & hsa-miR-616  & hsa-miR-466  \\
    hsa-miR-216b & hsa-miR-661  & hsa-miR-555  \\
    hsa-miR-498  & hsa-miR-920  & hsa-miR-603  \\
    hsa-miR-507  & hsa-miR-1323 & hsa-miR-622  \\
    hsa-miR-548  & hsa-miR-208a & hsa-miR-670  \\
                 & hsa-miR-374c & hsa-miR-892b \\
                 &              & hsa-miR-938  \\
                 &              & hsa-miR-216b \\
                 &              & hsa-miR-498  \\
                 &              & hsa-miR-507  \\
                 &              & hsa-miR-548  \\
                 &              & hsa-miR-1323 \\
                 &              & hsa-miR-208a \\
    \bottomrule
  \end{tabular}
\end{table}

\begin{figure}
\caption{The selected important miRs of the three measures}
\label{fig:venn}
\includegraphics[width=\textwidth]{{venn}.pdf}
\end{figure}


\section{Appendix}

\subsection{Parameters used in external libraries}

\subsubsection{GSEA-R}\label{src:gsea}

Program parameters:

\begin{verbatim}
  doc.string                  = "breast_cancer",
  non.interactive.run         = F,
  reshuffling.type            = "sample.labels",
  nperm                       = 1000,
  weighted.score.type         = 1,
  nom.p.val.threshold         = -1,
  fwer.p.val.threshold        = -1,
  fdr.q.val.threshold         = 0.25,
  topgs                       = 20,
  adjust.FDR.q.val            = F,
  gs.size.threshold.min       = 15,
  gs.size.threshold.max       = 500,
  reverse.sign                = F,
  preproc.type                = 0,
  random.seed                 = 111,
  perm.type                   = 0,
  fraction                    = 1.0,
  replace                     = F,
  save.intermediate.results   = F,
  OLD.GSEA                    = F,
  use.fast.enrichment.routine = T
\end{verbatim}

Analyzer parameters:

\begin{verbatim}
  directory = "results/",
  topgs     = 20,
  height    = 16,
  width     = 16
\end{verbatim}

\subsubsection{Random Forest R package}\label{src:rf}

\begin{verbatim}
  ntree       = 500
  sampsize    = if (replace) nrow(x) else ceiling(.632*nrow(x)),
  nodesize    = if (!is.null(y) && !is.factor(y)) 5 else 1,
  importance  = FALSE,
  localImp    = FALSE,
  nPerm       = 1,
  norm.votes  = TRUE,
  do.trace    = FALSE,
  keep.forest = !is.null(y) && is.null(xtest),
  corr.bias   = FALSE,
  keep.inbag  = FALSE
\end{verbatim}

\bibliography{mybib}{}
\bibliographystyle{plain}
\end{document}
